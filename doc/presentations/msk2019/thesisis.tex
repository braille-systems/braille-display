%XeLaTeX+makeIndex+BibTeX OR LuaLaTeX+...
\documentclass[a4paper,12pt]{article} %14pt - extarticle
\usepackage[utf8]{inputenc} %русский язык, не менять
\usepackage[T2A, T1]{fontenc} %русский язык, не менять
\usepackage[english, russian]{babel} %русский язык, не менять
\usepackage{fontspec} %различные шрифты
\setmainfont{Times New Roman}

%\defaultfontfeatures{Ligatures={TeX},Renderer=Basic}
\usepackage[hyphens]{url} %ссылки \url с переносами
\usepackage{hyperref} %гиперссылки href
\hypersetup{pdfstartview=FitH,  linkcolor=blue, urlcolor=blue, colorlinks=true} %гиперссылки
\usepackage{subfiles}%включение тех-текста
\usepackage{graphicx} %изображения
\usepackage{float}%картинки где угодно
\usepackage{textcomp}

\usepackage{dsfont}%мат. символы

\begin{document}
	
\title{Тренажёр Брайля: современные технологии для обучения незрячих. Тезисы}
\author{Валерий Зуев}

%------no date in title----------
%\date{\vspace{-5ex}} %for better solution - titling pkg
%--------------------------------
\maketitle

\section {Тезисы кратко}

\begin{enumerate}
	
	\item{} Типичные инновации в медицине/здравоохранении. Пример: датчик осанки.
	\item{} Тренажёр Брайля, его функции и устройство.
	\item{} Об истории разработки. Мой совет по планированию проектов (на примере тренажёра Брайля).
	\item{} Перспективы проекта. Увеличение числа ячеек. Машинное обучение для распознавания аудио и текста. Приложение под Android (вкратце).
	\item{} Заключение. В каких задачах уместно применять машинное обучение? (Там, где решение может быть выведено только из данных условий, притом много примеров ранее решённых задач).

\end{enumerate}

\section {Тезисы с описанием слайдов}

\begin{enumerate}
	
	\item{} Типичные инновации в медицине/здравоохранении. Пример: датчик осанки.
	\begin{enumerate}
		\item Титульный слайд. Символика Политеха, ФабЛаб
		\item Команда проекта
	\end{enumerate}

	\item{} Тренажёр Брайля, его функции и устройство.
	\begin{enumerate}
		\item Картинка с подписями: где ячейка, клавиатура, динамик, джойстик, кнопка помощи, кнопка режима звука
	\end{enumerate}
	
	\item{} Об истории разработки. Мой совет по планированию проектов (на примере тренажёра Брайля).
	\begin{enumerate}
		\item Круговой дисплей
		\item (Стройте блок-схему). Блок-схема. Лого Coursera, Stepik, Habr
		\item (Схема немного упрощена для наглядности). Прошлые версии. Евгений. Я не хочу сказать, что схема - универсальный рецепт. Но она может помочь вам не заузиться и не потратить много времени на путь, который не самый выгодный, потому что я видел много примеров, когда люди тратят уйму времени, иногда месяцы, годы, иногда всю жизнь кладут на алтарь своей идеи, когда можно было пойти иным путём, может, лишь чуть-чуть иным, а они в самом начале из множества возможных вариантов, которые не пришли в голову или пришли, но они забыли или убедили себя в том, что вариант не подходящий.
		Схемка позволяет следить, когда на что вы тратите свои силы. Любое планирование - не замена мыслительному процессу. Но, как голыми руками сложно создать 
	\end{enumerate}
	\item{} Перспективы проекта. Увеличение числа ячеек. Машинное обучение для распознавания аудио и текста. Приложение под Android (вкратце).
	\begin{enumerate}
		\item Брайль 3. Две картинки с рендерингом. (?) текущее состояние механики.
		\item Приложение под Андроид. Схематическая картинка: RPi и смартфон с буквой В и точками.
		\item Новый концепт. Картинка.
	\end{enumerate}
	
	\item{} Заключение. В каких задачах уместно применять машинное обучение? (Там, где решение может быть выведено только из данных условий, притом много примеров ранее решённых задач).
	\begin{enumerate}
		\item Задачи, в которых успешно применяется машинное обучение: классификация (выбор) и регрессия (оптимизация).
		\item Иллюстрация из книги Азимова
		\item Спасибо за внимание. Ссылки: braille, braille3, braille3parts
	\end{enumerate}
	
\end{enumerate}

\end{document}