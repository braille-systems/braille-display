%XeLaTeX+makeIndex+BibTeX OR LuaLaTeX+...
\documentclass[a4paper,12pt]{article} %14pt - extarticle
\usepackage[utf8]{inputenc} %русский язык, не менять
\usepackage[T2A, T1]{fontenc} %русский язык, не менять
\usepackage[english, russian]{babel} %русский язык, не менять
\usepackage{fontspec} %различные шрифты
\setmainfont{Times New Roman}

\usepackage{textcomp}
\begin{document}
	
\title{Тренажёр Брайля: современные технологии для обучения незрячих}
\author{Валерий Зуев}
\date{\vspace{-5ex}} %for better solution - titling pkg
\maketitle

Добрый день. Меня зовут Валерий Зуев, я студент Политехнического института. Уже три года в цифровой мастерской ФабЛаб Политех мы с товарищами проектируем Тренажёр Брайля - прибор для обучения незрячих чтению рельефным шрифтом.

Сначала разрешите мне показать людей, без которых не было бы приборов на первом слайде. А теперь я хочу вам кое-что рассказать.

Как вы считаете, из большого количества электронных и электромеханических устройств, которые в последнее время активно пытаются внедрить в здравоохранении, можно ли выделить какую-то группу наиболее успешно развивающихся приборов, которых лет пятнадцать назад нигде не было, а сейчас они повсюду? Это не экзоскелеты и не бионические протезы, точно не тренажёры Брайля. По моим наблюдениям, бурно развиваются различные датчики и системы датчиков.

Сенсоры температуры, давления, камеры, сканеры. Для незрячих - радары, предупреждающие о препятствиях вокруг. Системы контроля сравнительно легко конструировать и программировать, к тому же, от них редко требуется такая степень надёжности, как от тех же бионических протезов. Поэтому ими занимаются многие, от школьных команд до калифорнийских стартапов. Если вам интересна биология или медицина плюс хочется что-то изобрести, почему бы не начать с датчика? Сделайте какой-нибудь сенсор - может, для ветеринарии, или для проверки качества пищи - и принесите реальную пользу человечеству. Начните с датчика - это первая из трёх идей, которые я хотел сегодня высказать.

Есть электронные корректоры осанки. Помещаете под одежду небольшую коробочку. Если сутулитесь, он вибрирует. Однажды мне захотелось следить за своей осанкой, а датчика под рукой не было. Тогда я взял ненужную строительную рулетку, отрезал приличный кусок, сложил пополам и склеил скотчем. Получился "датчик осанки", который вставляется под одежду и, если я сутулюсь, с треском сгибается. Как видите, при решении любой проблемы полезно обдумать побольше вариантов, взвесить все "за" и "против", и не всегда наиболее выгодным оказывается высокотехнологичное решение. Переберите больше вариантов - это вторая из трёх идей, и сейчас мы её подробно обсудим на примере тренажёра Брайля.

Ещё в позапрошлом веке Луи Брайль изобрёл рельефный шрифт для незрячих, каждая буква - от одной до шести выпуклых точек. Воспринимаются на ощупь. В мире всё больше незрячих, и всё меньше грамотных по Брайлю. Рельефный шрифт выучить нелегко, и мы хотим сделать доступный самоучитель азбуки Брайля.

Задача: предоставить каждому незрячему возможность без посторонней помощи с нуля выучить азбуку Брайля. Наверное, здесь датчиками или механическими приспособлениями не обойтись. Мы хотели сделать подобие пошагового онлайн-курса, только вместо видео и текста основными каналами сделать голос и тактильное восприятие.

В России делают для обучения Брайлю прибор, который я бы по уровню отнёс к категории датчиков. Коробочка, на ней шесть кнопок. Набираете сочетание - прибор вслух произносит букву. Это скорее мини-справочник, чем самоучитель, он не развивает память пальцев. Есть дисплеи Брайля - приставки к компьютеру со строкой Брайля, которая может отобразить любой текст. Ключевой элемент дисплея - ячейки Брайля, их вы видите на слайде. Можно сделать обучающую программу для работы с таким дисплеем, но, к сожалению, дисплеи Брайля стоят от 150 000 рублей, бесплатно положены только слепоглухим людям и только после подтверждения знания азбуки Брайля. Существующие средства малопригодны, можно ли обойтись механикой - какой-нибудь набор табличек? Нет, это не похоже на онлайн-курс с шагами, заданиями, с проверкой. Так что здесь создание новой электромеханики выглядит оправданным.

Если кто-то читал Барбару Шер, мог узнать на предыдущем слайде схему обратного планирования, когда определяется цель, потом к ней придумываются способы достижения, потом ещё к ним - способы их достижения, и так пока не дойдёт до того, что можно начать делать прямо сейчас. Очень рекомендую использовать такие схемы в планировании, но если Вы - начинающий мейкер, разумно в начале пути нарисовать такую развилку: если есть сомнения, что делать, можно что-то изучить (почитать Хабр, пройти онлайн-курсы), а также пойти к эксперту (мы ходили в Общество Слепых). Помните, вторая идея, которую я прошу вас запомнить - придумать больше вариантов. Выпишите ваши варианты, составьте схему обратного планирования (в центре, где у меня маленькое пустое место, может быть на самом деле большое количество разных блоков и стрелочек) и принесите эксперту, пусть он посмотрит, что-то добавит, исправит, скажет, что изучить. Это может спасти как от недостатка идей, так и от избытка. Вот третья идея на сегодня.

Мы двинулись по пути электромеханики: изучили литературу и решили сделать доступную альтернативу дисплею Брайля. У американцев был патент на устройство, где ячейки помещались на боку вращающегося барабана, формируя бесконечную строку. Я разработал вариант, подходящий для 3Д-печати, но когда детали вышли из печати, стало понятно, что такая тонкая механика не будет надёжно работать.

Решено было умерить аппетиты. Последовала серия прототипов с одной-единственной ячейкой, этим долго занимался Глеб Андреевич Мирошник, магистр кафедры Теормеха. С такими книги не почитаешь, но память пальцев вырабатывается.

Вот прибор, который есть сейчас. Ячейка, особая клавиатура, джойстик. Научившись работе на таком, легко перейти на нормальный дисплей Брайля. Все электронные компоненты, в том числе моторы для ячейки, легко доступны и в сумме обойдутся менее чем в тысячу рублей. Остальные детали изготавливаются на 3Д-принтере и лазерном станке, все модели мы выложили в открытом доступе на GitHub, все программы тоже. 

Сейчас работаем над конструкцией с несколькими ячейками. Она будет более громоздкой и переключение символов будет не очень быстрым, но должно получиться дешевле дисплеев Брайля.

Тренажёр - это приставка к компьютеру, он соединяется по USB-кабелю, на компьютере размещается специальная программа с пошаговыми уроками, также заметки по Брайлю и прочее. Для тех, кто не совсем потерял зрение, уроки сопровождаюются картинками. В планах - подобное приложение для Android.

В уроках иногда предусмотрен голосовой ответ. Например, ячейка выводит букву, и надо её назвать вслух. Для распознавания голоса сейчас используется облачный сервис Google, то есть нужно интернет-соединение. Идёт работа над созданием собственного алгоритма машинного обучения для распознавания голосового ответа, чтобы не нужно было интернет-соединение.

\hrulefill


... Вначале, помните, я говорил о датчиках, следящих за здоровьем. Это - отличное место, где можно применить машинное обучение. Данные, собранные датчиком, можно передавать алгоритму, который, например, будет проверять наличие каких-то заболеваний.\\

\end{document}